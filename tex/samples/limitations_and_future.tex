Because it was difficult and expensive to recruit and interview highly-trained experts, this study relies on in-depth interviews with a small sample of humanists.
While such one-on-one interviews provided rich feedback, the opinions of our participants likely only approximate the true requirements of all historians and archivists. Moreover, interview studies may have limitations in unearthing design requirements (Section \ref{s:discussion_relevance}).
In the future, we thus plan to take steps to facilitate adoption in order to learn more about user needs.
In particular, we found that historians have to collect, organize, and sometimes digitize news stories before they are ready to gather and analyze query mentions (Section \ref{s:current_practices}). 
We thus plan to add features for importing news stories into \ours~from existing tools like Zotero and the Internet Archive. 
Additionally, throughout this work, we assume that query mentions are defined by exact string matches.
This simplifying assumption allows us to focus on user experience and interaction, but has clear limitations.
For instance, authors sometimes refer to ``Reagan'' using the nickname ``Dutch.''
Automatically detecting such aliases (and other deviations from exact string matching) will be important for future work.