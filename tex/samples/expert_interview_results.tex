Six themes emerged from qualitative coding, described below.

\subsection{\ours~helps with historical sensemaking}\label{s:sensemaking}

While using \ours, each of the five participants formed a question and then collected and interpreted evidence to start to answer that question.
We observed that \ours~helped historians with this investigative process, which Dalton and Charnigo \cite{DaltonCharnigo} describe as historical sensemaking. 
Our observations offered partial validation for our hypothesis that \ours~features can aid historians in their work (Section \ref{s:system}).

For instance, as part of his research, P1 studies \textit{New York Times} news coverage from journalists embedded with United States military units in the Iraqi city of Falluja during the second U.S.--Iraq war. 
From prior study, P1 understood that embedded U.S.\ journalists often published news stories reflecting the perspectives of U.S.\ military leaders. 
But while examining mentions of $Q$=``Falluja'' in \textit{New York Times} editorials using the \ours~interface, P1 expressed surprise when finding a more nuanced perspective from the opinion desk.  \textit{``I didn't see nearly as much of the sort of sensational depiction of Falluja, and the militants in Falluja [in editorials] that I expect from embedded journalists [in news stories],''} he reported. 

Similarly, P2 used~\ours~to find confirming evidence of shifting U.S.\ perspectives towards Robert Mugabe.
As P2 expected, early \textit{New York Times} editorials from the corpus praised $Q$=``Mugabe'' as a liberator, but then began to criticize \textit{``him as a bad statesman, as a tyrant and a dictator.''} 
P3 was likewise able to partially answer a research question with \ours.
She explained that while she had \textit{``a deep knowledge of [women in combat]. I don't have a deep knowledge of what the [NYT] editorial board has to say about it.''} 
Using \ours, she found evidence of editorials using \textit{``the gendered trope that women are supposed to be wives and mothers}.''
P5 also discovered an unexpected connection with musical copyright, while researching a hypothesis surrounding literary copyright. 
\textit{``The parity [with the music service] Napster that's that's really interesting ... That's not something I thought about ... I was thinking ... definitely more in literary items because that's what I deal with.''}

\subsection{\ours~features offer a corpus overview, alongside complementary context}\label{s:features_feedback}

Participants offered detailed feedback on \ours~features during interviews, which often matched our design goals for particular components of the interface.
To begin, three participants reported that \ours's \textbf{Time Series View} offered a useful overview of the entire corpus, by directing their attention to salient time periods. 
P1 said the Time Series View was an \textit{``easy way of visualizing''} corpus trends, and P5 suggested that the Time Series View might be helpful \textit{``when students are kind of in that exploratory phase ... as a way of ... coming up with research questions.''}
P4 offered similar feedback. 
\textit{``I really like this,''} she said. \textit{``This looks really functional and really useful. I like how there is quite a lot of information packed in.''} 

P1, P2 and P5 reported that the \textbf{Document Feed} was a useful feature of \ours~because it helped summarize query mentions. The Document Feed \textit{``condenses all of the essential information and sort of leaves out all the extra stuff},'' said P1. Said P2, \textit{``I found [the Document Feed] useful, especially the expand button. If I click expand I can see a rundown of the mentions right after the title without seeing the article}.'' 
P5 reported using the Document Feed to \textit{``do some ... simple kind of topic modeling in my own head ... just to see if I could pull out any ... themes there.''}
P5 added that, \textit{``having this here [i.e., Document Feed] is really helpful to kind of see what they're talking about.''}
P3 and P4 discussed the Document Feed while describing the importance of context in historical research; we include their feedback on this feature in Section \ref{s:feedback_context}. 

Several participants also reported that the \textbf{Document Viewer} helped during their research. For instance, P3 reported that automatic in-text highlighting in the feed was very helpful. \textit{``I'm a visual person. So I'm looking for the words. I like that they're in purple and green ... the words that you've given me the pop out ... and I can see if it's a pro or con article pretty quickly just from that}.'' P2 said he used the Document Viewer to \textit{``provide detail.''}

P1, P2 and P5 noted that \ours's linked \textbf{Document Feed and Document Viewer} served complementary purposes. 
They described how the Document Feed provided a summary of the query term, while the Document Viewer provided necessary and complementary details. \textit{``You need both [the Document Feed and Viewer]},'' said P2. \textit{``With just the Document Feed I won’t be able to get the full picture of the story. And with just the Document Viewer I will not be able to trace the mentions quite comprehensively and specifically}.'' P2 then added, \textit{``as a researcher, it’s important to see things in detail. If you just conclude from what you see in the Document Feed you are not going to get an objective picture of the context of the story line. But if you see the Document Feed, see the mentions, see what they imply, and then you want to understand the context of the story you are going to get to the Document Viewer}.''  
P1 said, \textit{``I like having both the Document Feed and the Document Viewer side by side. [The Document Viewer helps with] reading for more depth when I want more depth and [the Document Feed] helps with ... quick scans pretty easy.''}  
Similarly, P5 explained, \textit{``I see [the Document Feed and Viewer] working together really well ...
I start by looking at the feed to kind of pick out the articles that would want to kind of dive into deeper and then I go into the Document Viewer.''}

P4 and P5 specifically mentioned that complementary linked views from the Document Feed and Document Viewer helped with \textbf{\burdensome~and analysis}, as compared to a baseline \Baselongname~system. 
\textit{``A lot of a lot of databases that we work with do something similar to this [i.e., the Document Feed]},'' said P5, while describing a search engine results page. 
\textit{``But you often then have to click on the article to go into the article to get to that reading ... here it is nice that it was just kind of next to it and you can scroll through it.''} 
Similarly, P4 described the difficulties of context switching between documents from the Google search engine results page. 
\textit{``Obviously, it's is a time saver,''} she said, comparing \ours~to the \Baselongname~system. 
\textit{``You can tell ... just using the editorials at one newspaper.''}

Two participants relied on \ours's \textbf{filtering system} to investigate their research topics. P1 investigated the \textit{NYT} editorial board's discussion of the query term ``Falluja'' using the filter-by-subquery feature (e.g., searching for ``Falluja'' and ``resistance'' or ``Falluja'' and ``terrorist''). \textit{``It's pretty interesting to me that I get three hits with the words Falluja and resistance and only one with the word terrorist,''} he said. \textit{``That would suggest a certain orientation from the editorial board that will be unexpected}.'' P2 found the filter-by-count feature very helpful. \textit{``Oh, this is good},'' he said, while testing out the slider. \textit{``It gets us through to the most important, the most critical pieces that we want to read}.'' 

\subsection{Some disavow obligation to perform comprehensive review, noting high costs}\label{s:expert_interview_comprehensivenesss}

During needfinding, interviewees emphasized the importance of comprehensively reviewing all available evidence. 
However, to our surprise,  during the expert interview study, P4 explicitly disavowed an obligation to search comprehensively. \textit{``I don't feel like I have an obligation to look at everything,''} she said. \textit{``I have an obligation to get an overview and I think you know, with a completely unscientific measure of, oh, I think I've got enough now.''} 
Similarly, P1 commented that, 
\textit{``I don't think anyone actually does it [search comprehensively].''}
He went on \textit{``A lot of people pretend they do it ... [but] in terms of like visiting archives ... everyone's skimming ... they already know what they're looking for and they're just trying to find it.''}
P2 pointed out that comprehensive manual review was desirable but ultimately had high costs. \textit{``I am not saying we should get rid of personal scrutiny, the way you do it yourself. [But] you want to save time. If you do it [i.e., read] one-by-one it wastes too much time}.'' We discuss ambiguity surrounding comprehensive review in Section \ref{s:discussion}.

\subsection{Context is crucial in historical research, so some are wary of text summarization}\label{s:feedback_context}

Like during needfinding (Section \ref{s:needs_context}), participants often emphasized the importance of context in historical research. For instance, P3 described extensive research to prepare for oral history interviews in order to \textit{``get that context to be able to ask them the questions that I asked them}.''  P2 also reported that context is \textit{``very important''} for historians, as it \textit{``helps you understand why things are what they are.''} 

Some historians' emphasis on context informed their feedback on the Document Feed. While P1, P2 and P5 found the Document Feed useful (Section \ref{s:features_feedback}), P3 and P4 expressed reservations because they felt they needed more context to reach conclusions. P3 took the more extreme position.
\textit{``For me, I don't know if [the Document Feed] is necessary},'' she said. \textit{``As a history scholar, you can't take things out of context. You need to know the bigger context.''} 
On the other hand, P4 reported that she would need more context (i.e., longer extractions from news stories) before the feature would be useful. 
\textit{``The more context I can take in within as compact a time frame and compact a format, but sufficiently informative [the better]''} she said. \textit{``But I think these [shortened sentences in the Document Feed] might have to be longer for that to work}.''  

\subsection[Tradeoffs between neutral review and limited time]{Some users recognize a tradeoff between neutral review and limited time}\label{s:relevance_model_feedback}

During needfinding and prototyping, interviewees often stressed the importance of avoiding possible bias from software in historical research.
But during our expert interview study, P4 reported that she relied on black-box relevance models to direct her attention while searching archives.
\textit{``I do try to use the chronological sorting [when using ProQuest],''} said P4. \textit{``But it is ... too much to wade through. If your corpus is reasonably big then you have to have a relevance kind of algorithm in there.
Otherwise, it's just going to be too frustrating.''} 
P4 also recognized that reliance on ranking introduces confounds. 
\textit{``I think it would be appropriate to make people look at all of the irrelevant stuff},'' she said. 
\textit{``So they realize the algorithm is pulling the relevant stuff for you ...
but you can’t make the search s*** for people just to sort of make that point.''} 

On the other hand, P5 liked how \ours~used filtering to avoid potential bias. \textit{``I think it's better that its just showing everything,''} he explained. \textit{``I prefer having everything there to kind of whittle down ... as opposed to having certain things like cherry-picked ... I guess it's never super clear to me why certain things might be moved to the top of results ... it raises questions about how things are ordered and how they're brought to light.''}

As \ithree~predicted (Section \ref{s:needs_trust}), P1 described relying on the search function of the \textit{New York Times} website \cite{nytwebsite}, without understanding how the site was ranking search results by relevance. 
\textit{``I wasn't super aware of how they were pulling up articles for me ... They rank it in terms of views right?''} he said. 
He added, \textit{``I just don't, you know, have the knowledge of how to navigate these ... search engines well enough.''}
We discuss mixed feedback on algorithmic bias in Section \ref{s:discussion_relevance}.

\subsection{Access, integrity and integration are important to current practices}\label{s:current_practices}

Many participants commented on the importance of access, integrity and integration in describing their current practices with newspaper archives (see also Section \ref{s:limits_and_future}).
{P1} reported gathering news articles on U.S.-Iraqi relations from around the web ``for years'' by using search engines like Google or the \textit{New York Times} website \cite{nytwebsite}, saving these articles to the Internet Archive \cite{InternetArchive}, and then organizing this collection using the software program Omeka \cite{omeka}. This participant pointed out that \ours~\textit{``assumes you have found all the stuff you want to work with,''} which is not true for his current research. 
{P2} said that he had to rely on physical archives of print newspapers in Zimbabwe, which required burdensome international travel. 
{P3} said that she rarely used newspapers in her own research because many newspaper archives are often inaccessible behind paywalls, and {P4} emphasized the need for better optical character recognition technology to improve search over printed newspapers.
{P5} reported that he \textit{``used Zotero a lot''} to store and organize archival sources; he liked that Zotero is open source and integrates with Microsoft Word.
